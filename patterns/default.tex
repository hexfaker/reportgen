%Description: report output file form
%Author:  Raudik Sergei (Raudik_sa@mail.ru)
%Created at: Wed Feb 25 22:22 MSK 2015

%% ---will be inserted by a script--
<[class]>
%\documentclass[12pt,a4paper]{article}
%% --------
\usepackage[utf8]{inputenc}
%% --babel--
<[babel]>
%\usepackage[english,russian]{babel}
%% ------
\usepackage[OT1]{fontenc}
%% --math--
<[math]>
%\usepackage{amsmath}
%\usepackage{amsfonts}
%\usepackage{amssymb}
%% -------
\usepackage{graphicx}
\usepackage{multirow}
\usepackage[left=2cm,right=2cm,top=2cm,bottom=2cm]{geometry}

%% ---will be inserted by a script--
%%\title{Лабораторная работа №"nn"\\"Title"}
<[title]>
%%\author{"Human"}
<[author]>
%% --------

\begin{document}
\maketitle

\begin{tabular}{p{12cm} l}
\textit{Цель работы:\newline} "" \newline \newline
\textit{Оборудование:\newline}  
\end{tabular}\\ \\

\textit{Ход Работы:}
\begin{enumerate}
  \item 
  \item 
\end{enumerate}

\textit{Cхема установки}
<[scheme]>
\textit{Результаты:\\}
<[table]>

\textit{Графики:\\}
<[img]>

\textit{Формулы:}
\begin{align*}

\end{align*}
\textit{Формулы погрешностей:}
\begin{align*}
  & \sigma_x = \sqrt{\frac{\sum\limits_{i=0}^{N}x_i^2}{N}} \\
  & \sigma = \frac{\sigma_x}{N-1};% N in other variants 
\end{align*}

\textit{Выводы:\\}
\begin{enumerate}
  \item 
\end{enumerate}
\end{document}
